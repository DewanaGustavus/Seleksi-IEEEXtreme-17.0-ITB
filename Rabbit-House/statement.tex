\documentclass{article}

\usepackage{geometry}
\usepackage{amsmath}
\usepackage{graphicx, eso-pic}
\usepackage{listings}
\usepackage{hyperref}
\usepackage{multicol}
\usepackage{fancyhdr}
\pagestyle{fancy}
\fancyhf{}
\hypersetup{ colorlinks=true, linkcolor=black, filecolor=magenta, urlcolor=cyan}
\geometry{ a4paper, total={170mm,257mm}, top=10mm, right=20mm, bottom=20mm, left=20mm}
\setlength{\parindent}{0pt}
\setlength{\parskip}{0.3em}
\renewcommand{\headrulewidth}{0pt}

\rfoot{\thepage}
\fancyhf{} % sets both header and footer to nothing
\renewcommand{\headrulewidth}{0pt}
\lfoot{\textbf{Seleksi IEEEXtreme 17.0 ITB}}
\pagenumbering{gobble}

\fancyfoot[CE,CO]{\thepage}
\lstset{
    basicstyle=\ttfamily\small,
    columns=fixed,
    extendedchars=true,
    breaklines=true,
    tabsize=2,
    prebreak=\raisebox{0ex}[0ex][0ex]{\ensuremath{\hookleftarrow}},
    frame=none,
    showtabs=false,
    showspaces=false,
    showstringspaces=false,
    prebreak={},
    keywordstyle=\color[rgb]{0.627,0.126,0.941},
    commentstyle=\color[rgb]{0.133,0.545,0.133},
    stringstyle=\color[rgb]{01,0,0},
    captionpos=t,
    escapeinside={(\%}{\%)}
}

\begin{document}

\begin{center}

    
    \section*{Rabbit House} % ganti judul soal

    \begin{tabular}{ | c c | }
        \hline
        Batas Waktu  & 1s \\    % jangan lupa ganti time limit
        Batas Memori & 256MB \\  % jangan lupa ganti memory limit
        \hline
    \end{tabular}
\end{center}

\subsection*{Deskripsi}

Kafuu Chino merupakan seorang penerus sebuah kafe terkenal yang bernama Rabbit House. Demi menjaga nama baik kafe miliknya Chino ingin meningkatkan kualitas kopi yang dijualnya. Untuk meningkatkan kualitas kafenya, Chino meminta Cocoa yang merupakan pekerja paruh baya di kafenya untuk men survei ketertarikan para pelanggan dengan kopi yang tersedia pada Rabbit House. Cocoa dengan cepat memenuhi permintaan Chino, Ia men survei N buah pelanggan untuk memberikan sebuah penilaian untuk sebuah kopi yang akan diberikan.\\

Penilaian yang menjadi acuan bagi Rabbit House adalah sebagai berikut, Rabbit House menjual K buah kopi berbeda yang nantinya masing-masing pelanggan akan memberikan penilaian untuk sebuah kopi yang diminum. Lalu untuk sebuah kopi akan memiliki kontribusi nilai berupa kuadrat dari jumlah penilaian pelanggan untuk kopi tersebut. Chino percaya bahwa kualitas kafe yang Ia miliki dinilai berdasarkan total dari nilai seluruh kopi.\\

Setelah mencatat seluruh penilaian pelanggan Cocoa lupa untuk mencatat kopi apa yang dinilai oleh setiap pelanggan dan hanya mencatat penailaiannya saja. Karena tidak ingin memarahi Cocoa yang sedang merasa bersalah, Chino tetap menggunakan survei yang didapat Cocoa dan memutuskan untuk memakai perhitungan nilai harapan kualitas kafenya sebagai acuan penilaian. Bantulah Chino untuk menghitung nilai harapan kualitas kafe miliknya.\\

Untuk format keluaran semisal nilai harapan kualitas kafe Rabbit House direpresentasikan dengan pecahan sederhana $\frac{P}{Q}$ dengan Q relatif prima dengan 998244353. Keluarkan sebuah nilai bilangan bulat $P \times Q^{-1}$ modulo 998244353. Dimana $Q^{-1}$ modulo 998244353 adalah nilai yang memenuhi persamaan $Q^{-1} \times Q$ $\equiv 1$ (mod 998244353). 

\subsection*{Format Masukan}

Baris pertama terdiri dari dua bilangan bulat positif $N, K$ ($1 \leq N \leq 10, 1 \leq K \leq 10^{5}$), yang merupakan banyaknya pelanggan yang di survei oleh Cocoa dan banyaknya kopi yang dijual di Rabbit House.\\
Baris kedua terdiri dari N buah bilangan bulat positif $R_i$ ($1 \leq R_i \leq 10^9, 1 \leq i \leq N $), yang merupakan penilaian pelanggan ke-i.

\subsection*{Format Keluaran}

Keluarkan sebuah bilangan bulat positif yang merupakan nilai harapan dari kualitas kafe Rabbit House.

\begin{multicols}{2}
\subsection*{Contoh Masukan 1}
\begin{lstlisting}
5 1
1 2 3 4 5
\end{lstlisting}
\columnbreak
\subsection*{Contoh Keluaran 1}
\begin{lstlisting}
225
\end{lstlisting}
\vfill
\null
\end{multicols}

\begin{multicols}{2}
\subsection*{Contoh Masukan 2}
\begin{lstlisting}
3 4
1 2 3
\end{lstlisting}
\columnbreak
\subsection*{Contoh Keluaran 2}
\begin{lstlisting}
499122196
\end{lstlisting}
\vfill
\null
\end{multicols}

\subsection*{Penjelasan}
Pada contoh pertama, hanya terdapat 1 kopi jadi seluruh pelanggan akan menilai kopi yang sama sehingga total nilai untuk kopi 1 adalah total penilaian pelanggan yaitu 15. Untuk nilai kualitas kafe Chino adalah kuadrat dari nilai kopi sehingga nilai kualitas kafe Chino adalah $15^2$ = 225\\

Pada contoh kedua, terdapat 3 pelanggan dan 4 jenis kopi, semisal kopi yang diberikan kepada pelanggan adalah\\
Pelanggan 1 : kopi 1\\
Pelanggan 2 : kopi 1\\
Pelanggan 3 : kopi 3\\
penilaian untuk setiap kopi adalah\\
Kopi 1 : 3\\
Kopi 2 : 0\\
Kopi 3 : 3\\
Kopi 4 : 0\\
nilai kualitas kafe Chino untuk kasus ini adalah $3^2 + 0^2 + 3^2 + 0^2$ = 18\\
nilai harapan kualitas kafe adalah $\frac{P}{Q}$ dimana P adalah jumlah dari nilai kualitas kafe untuk seluruh kemungkinan dan Q adalah jumlah kemungkinan yang mungkin, apabila dihitung didapat bahwa nilai harapan kualitas kafe adalah $\frac{1248}{64}$, lalu dapat dihitung bahwa $P \times Q^{-1} \equiv 499122196$ (mod 998244353).

\end{document}