\documentclass{article}

\usepackage{geometry}
\usepackage{amsmath}
\usepackage{graphicx, eso-pic}
\usepackage{listings}
\usepackage{hyperref}
\usepackage{multicol}
\usepackage{fancyhdr}
\pagestyle{fancy}
\fancyhf{}
\hypersetup{ colorlinks=true, linkcolor=black, filecolor=magenta, urlcolor=cyan}
\geometry{ a4paper, total={170mm,257mm}, top=10mm, right=20mm, bottom=20mm, left=20mm}
\setlength{\parindent}{0pt}
\setlength{\parskip}{0.3em}
\renewcommand{\headrulewidth}{0pt}

\rfoot{\thepage}
\fancyhf{} % sets both header and footer to nothing
\renewcommand{\headrulewidth}{0pt}
\lfoot{\textbf{Seleksi IEEEXtreme 17.0 ITB}}
\pagenumbering{gobble}

\fancyfoot[CE,CO]{\thepage}
\lstset{
    basicstyle=\ttfamily\small,
    columns=fixed,
    extendedchars=true,
    breaklines=true,
    tabsize=2,
    prebreak=\raisebox{0ex}[0ex][0ex]{\ensuremath{\hookleftarrow}},
    frame=none,
    showtabs=false,
    showspaces=false,
    showstringspaces=false,
    prebreak={},
    keywordstyle=\color[rgb]{0.627,0.126,0.941},
    commentstyle=\color[rgb]{0.133,0.545,0.133},
    stringstyle=\color[rgb]{01,0,0},
    captionpos=t,
    escapeinside={(\%}{\%)}
}

\begin{document}

\begin{center}

    
    \section*{Formasi Tentara} % ganti judul soal

    \begin{tabular}{ | c c | }
        \hline
        Batas Waktu  & 1s \\    % jangan lupa ganti time limit
        Batas Memori & 256MB \\  % jangan lupa ganti memory limit
        \hline
    \end{tabular}
\end{center}

\subsection*{Deskripsi}

Renkawa Cherino adalah seorang pemimpin pasukan tentara Red Winter. Pasukan Red Winter terdiri atas 2 tipe tentara yaitu tipe bertahan dan tipe penyerang. Cherino ingin membuat formasi pasukan dengan bentuk formasi N x N tentara terdiri atas pasukan bertahan atau penyerang. Namun penyusunan formasi tidak boleh sembarang sebab dapat menjadi celah bagi musuh untuk dapat mengalahkan pasukan Red Winter. Untuk mencegah celah pertahanan, Cherino memberi M buah batasan pada formasi. Pada sebuah batasan, Cherino ingin memastikan bahwa pada sebuah subformasi yaitu untuk sebuah area, susunan tentara seimbang yaitu perbedaan antara banyak tentara bertahan dan banyak tentara penyerang hanya berbeda maksimal 1 anggota. Bantu Cherino membuat formasi yang memenuhi batasan sehingga formasi tersebut tidak dapat dikalahkan dengan mudah oleh musuh.

\subsection*{Format Masukan}

Baris pertama terdiri dari dua bilangan bulat positif $N, M (1 \leq N \leq 10^3, 1 \leq M \leq 10^3)$, yang merupakan banyaknya baris sekaligus banyaknya kolom pada formasi pasukan Red Winter, dan banyaknya batasan yang diberikan Cherino.\\

M baris selanjutnya terdiri dari empat bilangan bulat positif $x_1, y_1, x_2, y_2 (1 \leq x_1 \leq x_2 \leq N, 1 \leq y_1 \leq y_2 \leq N)$, yang merupakan area persegi panjang yang diberi batasan oleh Cherino dengan $(x_1, y_1)$ menjadi titik kiri atas persegi panjang dan $(x_2, y_2)$ menjadi titik kanan bawah persegi panjang.

\subsection*{Format Keluaran}

Berikan keluaran N baris dengan setiap baris berisi N buah bilangan 0 atau 1 dipisahkan spasi dimana 0 merepresentasikan tentara bertahan dan 1 merepresentasikan tentara penyerang. Serta keluaran harus dipastikan memenuhi batasan-batasan yang diberikan oleh Cherino.

\begin{multicols}{2}
\subsection*{Contoh Masukan}
\begin{lstlisting}
5 3
1 1 1 1
1 2 4 3
3 3 5 5
\end{lstlisting}
\columnbreak
\subsection*{Contoh Keluaran}
\begin{lstlisting}
0 0 0 0 0
1 1 1 1 1
0 0 0 0 1
1 1 1 1 1
0 0 0 0 0
\end{lstlisting}
\vfill
\null
\end{multicols}

\subsection*{Penjelasan}
Pada contoh kasus, kita diminta untuk membuat formasi 5 x 5.\\
Pada batasan pertama terbentuk subformasi\\
0\\
Jumlah tentara bertahan adalah 1, dan jumlah tentara penyerang adalah 0, sehingga memenuhi batasan karena perbedaannya tidak lebih dari 1.\\
    
Pada batasan kedua terbentuk subformasi\\
1 1 1 1\\
0 0 0 0\\
Jumlah tentara bertahan adalah 4, dan jumlah tentara penyerang adalah 4, sehingga memenuhi batasan karena perbedaannya tidak lebih dari 1.\\
    
Pada batasan ketiga terbentuk subformasi\\
0 0 1\\
1 1 1\\
0 0 0\\
Jumlah tentara bertahan adalah 5, dan jumlah tentara penyerang adalah 4, sehingga memenuhi batasan karena perbedaannya tidak lebih dari 1.\\

Contoh batasan dimana subformasi tidak memenuhi adalah subformasi (4, 2) sampai (5, 3) yang akan membentuk subformasi\\
1 1\\
0 1\\
Jumlah tentara bertahan adalah 1, sedangkan jumlah tentara penyerang adalah 3, sehingga tidak memenuhi batasan karena perbedaannya lebih dari 1.\\

    
\end{document}