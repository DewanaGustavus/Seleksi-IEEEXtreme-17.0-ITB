\documentclass{article}

\usepackage{geometry}
\usepackage{amsmath}
\usepackage{graphicx, eso-pic}
\usepackage{listings}
\usepackage{hyperref}
\usepackage{multicol}
\usepackage{fancyhdr}
\pagestyle{fancy}
\fancyhf{}
\hypersetup{ colorlinks=true, linkcolor=black, filecolor=magenta, urlcolor=cyan}
\geometry{ a4paper, total={170mm,257mm}, top=10mm, right=20mm, bottom=20mm, left=20mm}
\setlength{\parindent}{0pt}
\setlength{\parskip}{0.3em}
\renewcommand{\headrulewidth}{0pt}

\rfoot{\thepage}
\fancyhf{} % sets both header and footer to nothing
\renewcommand{\headrulewidth}{0pt}
\lfoot{\textbf{Seleksi IEEEXtreme 17.0 ITB}}
\pagenumbering{gobble}

\fancyfoot[CE,CO]{\thepage}
\lstset{
    basicstyle=\ttfamily\small,
    columns=fixed,
    extendedchars=true,
    breaklines=true,
    tabsize=2,
    prebreak=\raisebox{0ex}[0ex][0ex]{\ensuremath{\hookleftarrow}},
    frame=none,
    showtabs=false,
    showspaces=false,
    showstringspaces=false,
    prebreak={},
    keywordstyle=\color[rgb]{0.627,0.126,0.941},
    commentstyle=\color[rgb]{0.133,0.545,0.133},
    stringstyle=\color[rgb]{01,0,0},
    captionpos=t,
    escapeinside={(\%}{\%)}
}

\begin{document}

\begin{center}

    
    \section*{Abi-Eshuh} % ganti judul soal

    \begin{tabular}{ | c c | }
        \hline
        Batas Waktu  & 1s \\    % jangan lupa ganti time limit
        Batas Memori & 256MB \\  % jangan lupa ganti memory limit
        \hline
    \end{tabular}
\end{center}

\subsection*{Deskripsi}

Asuma Toki adalah seorang siswi dari SMA Millennium yang tergabung dalam klub C\&C. Sebagai anggota dari C\&C yang rajin, Toki berusaha sangat keras untuk menjadi perfect maid dan akan menerima seluruh misi yang diberikan kepadanya. Ketua OSIS di SMA Millennium yaitu Rio sedang membutuhkan bantuan dari C\&C. Terdapat sebuah AI misterius bernama Divi:SION yang dapat menciptakan banyak robot dan berencana untuk menghancurkan kota Kivotos. Rio mempunyai Q buah permintaan yang berbeda dimana untuk sebuah permintaan yang dibutuhkan Rio, Rio meminta bantuan anggota C\&C untuk membasmi X buah robot jahat buatan Divi:SION dengan waktu yang sesingkat mungkin.\\

Beruntung, Toki memiliki sebuah sistem alat tempur mutakhir yang bernama Abi-Eshuh. Abi-Eshuh bekerja dengan sistem sebagai berikut, pada armor Abi-Eshuh akan terdapat sebuah level kekuatan laser yaitu K. Untuk setiap detik, Toki akan menekan salah satu dari dua tombol berikut:
\begin{enumerate}
    \item Tombol A : Abi-Eshuh akan menembak K buah laser yang akan membasmi K buah robot 
    \item Tombol B : Abi-Eshuh akan menembak K buah laser seperti tombol A, dan kekuatan laser Abi-Eshuh akan bertambah sebesar 1
\end{enumerate}
Perlu diperhatikan bahwa Toki harus berhati-hati dalam menembakkan laser, karena apabila sudah tidak ada robot dan Toki menembakkan laser, maka laser tersebut tidak akan mengenai musuh dan akhirnya akan mengenai dan menghancurkan kota. Selain itu untuk membantu pekerjaan Toki, Rio memberikan dana untuk mengupgrade armor Abi-Eshuh menjadi berkekuatan K sebelum Toki melaksanakan tugasnya. Bantulah Toki untuk menentukan waktu minimum untuk menyelesaikan permintaan yang diberikan oleh Rio. Perhatikan juga bahwa bisa saja permintaan Rio menjadi mustahil untuk dilakukan karena kekuatan laser Abi-Eshuh yang terlalu besar dan justru akan menghancurkan kota apabila dipaksakan, untuk kasus seperti itu keluarkan nilai -1.

\subsection*{Format Masukan}

Baris pertama terdiri dari satu bilangan bulat positif $Q$ ($1 \leq Q \leq 100.000$), yang menyatakan banyaknya bantuan yang dibutuhkan Rio.\\
Q baris selanjutnya terdiri dari dua bilangan bulat positif $X$ dan $K$ ($1 \leq X, K \leq 10^{18}$), yang merupakan banyaknya robot yang perlu dibasmi oleh Toki dan nilai awal kekuatan laser Abi-Eshuh.

\subsection*{Format Keluaran}

Untuk setiap bantuan yang dibutuhkan Rio, keluarkan sebuah bilangan bulat yang merupakan waktu minimum bagi Toki untuk membasmi robot-robot yang diminta Rio atau keluarkan -1 apabila permintaan mustahil untuk dilakukan.

\begin{multicols}{2}
\subsection*{Contoh Masukan}
\begin{lstlisting}
2
10 6
100 5
\end{lstlisting}
\columnbreak
\subsection*{Contoh Keluaran}
\begin{lstlisting}
-1
11
\end{lstlisting}
\vfill
\null
\end{multicols}

\subsection*{Penjelasan}
Pada permintaan pertama mustahil bagi Toki untuk membasmi 10 buah robot dengan kekuatan awal 6, karena apapun tombol yang Toki tekan, jumlah robot akan berkurang 6 dan menjadi 4 buah. Lalu setelah itu Toki tidak dapat menekan tombol karena kekuatan laser melebihi jumlah robot yang kalau dipaksakan malah akan menghancurkan kota.

\pagebreak
Pada permintaan kedua misalkan urutan penekanan tombol oleh Toki adalah ABAABBBAABAAA, maka urutan kondisi yang terjadi adalah sebagai berikut : \\
\begin{tabular}{ | c | c | c | c | }
    \hline
    Detik & Tombol & Kekuatan & Robot \\ 
    \hline
    0 &  & 5 & 100\\  
    \hline
    1 & A & 5 & 95\\  
    \hline
    2 & B & 6 & 90\\  
    \hline
    3 & A & 6 & 84\\  
    \hline
    4 & A & 6 & 78\\  
    \hline
    5 & B & 7 & 72\\  
    \hline
    6 & B & 8 & 65\\  
    \hline
    7 & B & 9 & 57\\  
    \hline
    8 & A & 9 & 48\\  
    \hline
    9 & A & 9 & 39\\  
    \hline
    10 & B & 10 & 30\\  
    \hline
    11 & A & 10 & 20\\  
    \hline
    12 & A & 10 & 10\\  
    \hline
    13 & A & 10 & 0\\  
    \hline
\end{tabular}
\\
\\
Perlu diperhatikan bahwa urutan ini bukan solusi yang optimal.

\pagebreak

\end{document}